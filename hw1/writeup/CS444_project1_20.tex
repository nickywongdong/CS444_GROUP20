\documentclass[onecolumn, draftclsnofoot,10pt, compsoc]{IEEEtran}
\usepackage{graphicx}
\usepackage{url}
\usepackage{setspace}
\usepackage{tipa}
\usepackage{array, longtable}
\usepackage{listings}%
\usepackage{color}
\usepackage[usenames,dvipsnames,svgnames,table]{xcolor}
\lstset{%
  basicstyle=\ttfamily\footnotesize,
  backgroundcolor=\color{gray!10},%
  tabsize=3,
  language=bash,
  columns=fullflexible,
  %frame=shadowbox,
  frame=leftline,
  rulesepcolor=\color{gray},
  xleftmargin=20pt,
  framexleftmargin=15pt,
  keywordstyle=\color{black}\bf,
  commentstyle=\color{OliveGreen},
  %stringstyle=\color{red},
  breaklines=true,
  %numbers=left,
  %numberstyle=\tiny,
  %numbersep=5pt,
  showstringspaces=false,
  postbreak=\mbox{{/}\space},
  emph={str},emphstyle={\color{magenta}},
  alsoletter=-,%to allow qemu-system...
  %add extra keywords here.
  emph={%  
    cd, cp, qemu-system-i386, mv, mkdir, make, for, git, gdb, uname, tar, chmod%
    },emphstyle={\color{black}\bf},
  %remove keywords below
  deletekeywords={enable, continue}
}%


\usepackage{geometry}
\geometry{textheight=9.5in, textwidth=7in}

\usepackage{graphicx}
\graphicspath{{images/}}

% 1. Fill in these details
\def \TeamName{   Group 20}
\def \GroupMemberOne{     Max Moulds}
\def \GroupMemberTwo{     Monica Sek}
\def \GroupMemberThree{     Nicholas Wong}
\def \ProfessorPerson{    Dr. Kevin McGrath}


% 2. Uncomment the appropriate line below so that the document type works
\def \DocType{    HW1 Writeup
        }

\newcommand{\NameSigPair}[1]{\par
\makebox[2.75in][r]{#1} \hfil   \makebox[3.25in]{\makebox[2.25in]{\hrulefill} \hfill    \makebox[.75in]{\hrulefill}}
\par\vspace{-12pt} \textit{\tiny\noindent
\makebox[2.75in]{} \hfil    \makebox[3.25in]{\makebox[2.25in][r]{Signature} \hfill  \makebox[.75in][r]{Date}}}}
% 3. If the document is not to be signed, uncomment the RENEWcommand below
%\renewcommand{\NameSigPair}[1]{#1}

%%%%%%%%%%%%%%%%%%%%%%%%%%%%%%%%%%%%%%%
\begin{document}
\begin{titlepage}
    \pagenumbering{gobble}
    \begin{singlespace}
        \hfill
        \par\vspace{.2in}
        \centering
        \scshape{
            \huge Operating Systems II \DocType \par
            {\large\today}\par
            \vspace{.5in}
            \vfill
            \vspace{5pt}
            %{\Large\NameSigPair{\ProfessorPerson}\par}
            {\large Prepared for }\par
            \ProfessorPerson\par
            {\large Prepared by }\par
            \GroupMemberOne\par
            \GroupMemberTwo\par
            \GroupMemberThree\par
            \vspace{5pt}
            \vspace{20pt}
        }
        \begin{abstract}
        %6. Fill in your abstract
          This document is the writeup for homework 1 of Operating Systems II Spring term 2018, written by Group 20.
          \end{abstract}
    \end{singlespace}
\end{titlepage}
\newpage
\pagenumbering{arabic}
\tableofcontents
% 7. uncomment this (if applicable). Consider adding a page break.
%\listoffigures
%\listoftables
\clearpage

\section{Log of Commands}
The directory for the group was created by navigating to the proper directory, followed by a mkdir. Permissions were granted so the whole group could access this directory:
\begin{lstlisting}
cd /scratch/spring2018
mkdir group20
chmod -R 777 group20
\end{lstlisting}

We then sourced the environment using:
\begin{lstlisting}
source /scratch/files/environment-setup-i586-poky-linux.csh
\end{lstlisting}

The kernel image was pulled from the supplied repository and unpacked:
\begin{lstlisting}
tar xvjf linux-yocto-3.19.2.tar.bz2
\end{lstlisting}

We then copied the configuration file, and placed it in the root of the linux tree, then renamed it to .config:
\begin{lstlisting}
cd /scratch/spring2018/group20/linux-yocto-3.19.2
cp /scratch/files/config-3.19.2-yocto-standard .
mv config-3.19.2-yocto-standard .config
\end{lstlisting}

The kernel was then built using 4 threads:
\begin{lstlisting}
make -j4 all
\end{lstlisting}

The VM was then ran with the following command:
\begin{lstlisting}
qemu-system-i386 -gdb tcp::5520 -S -nographic -kernel linux-yocto-3.19.2/arch/i386/boot/bZimage -drive file=core-image-lsb-sdk-qemux86.ext4,if=virtio -enable-kvm -net none -usb -localtime --no-reboot --append "root=/dev/vda rw console=ttyS0 debug"
\end{lstlisting}

GDB was used to continue the process:
\begin{lstlisting}
gdb -tui
    remote target localhost:5520
    continue
\end{lstlisting}

The following was used to verify that the kernel was indeed the custom kernel that was just built:
\begin{lstlisting}
uname -a
    Linux qemux86 3.19.2Group20Makefile-group2-config-file+ #1 SMP PREEMPT Thu Apr 12 09:02:01 PDT 2018 i686 GNU/Linux
\end{lstlisting}
Compared to the provided kernel:
\begin{lstlisting}
uname -a
    Linux qemux86 3.19.2-yocto-standard #1 SMP PREEMPT Wed Apr 1 00:11:06 GMT 2015 i686 GNU/Linux
\end{lstlisting}

\newpage
\section{qemu Flags and Explanations}
Flags:
\begin{itemize}
    \item -gdb tcp::5520:
    
        Waits for gdb tcp connection
    \item -S:
    
        Shorthand for -gdb tcp::5520, opens gdbserver on TCP port 5520
    \item -nographic:
    
        Disables graphical output
    \item -kernel linux-yocto-3.19.2/arch/i386/boot/bzimage
    
        Selects bzimage as the kernel to load
    \item -drive file=core-image-lsb-sdk-qemux86.ext4,if=virtio -enable-kvm
    
        Defines a new virtual drive, and uses Virtio and enables KVM if available.
    \item -net none
    
        Disables network
    \item -usb
    
        Enables USB driver
    \item -localtime
    
    \item --no-reboot
    
        Specify VM to exit instead of rebooting
    \item --append "root=/dev/vda rw console=ttys0 debug"
    
        give kernel command line arguments to specify root directory, and to use ttys0 as debug console
\end{itemize}

\newpage
\section{Version Control Log}

%% uncomment this when you have the file on machine
%%\begin{longtable} {|l|l|l|}
    \hline
    \textbf{Author} & \textbf{Date} & \textbf{Message} \ \hline
nickywongdong & 2018-04-08 & first commit \\ \hline
nickywongdong & 2018-04-10 & copying and unpacking yocto 3.19.2 image \\ \hline
nickywongdong & 2018-04-10 & copying .ext4 driver file using tar, and adding gitignore for the file \\ \hline
nickywongdong & 2018-04-10 & adding a gitignore in the hw directory as well \\ \hline
nickywongdong & 2018-04-12 & unpacking initial kernel from repo \\ \hline
nickywongdong & 2018-04-12 & adding custom version tag to Makefile under extra version \\ \hline
nickywongdong & 2018-04-12 & copying config from /scratch/files and renaming to .config, then running make menuconfig (don't know if we need this step or not) \\ \hline
nickywongdong & 2018-04-12 & moving into one git repo after build of final kernel. Adding initial state of writeup for latex \\ \hline
nickywongdong & 2018-04-12 & attempt to merge intial git with new repo \\ \hline
nickywongdong & 2018-04-12 & moving kernel outside of hw directories \\ \hline
nickywongdong & 2018-04-12 & finalizing moving kernel root tree outside of hw directory \\ \hline
nickywongdong & 2018-04-12 & adding git log latex table generator \\ \hline
nickywongdong & 2018-04-12 & changing git log table generator, previous required certain installation and rights \\ \hline
nickywongdong & 2018-04-12 & updating gitignore, removing previous latex-git-log \\ \hline
nickywongdong & 2018-04-12 & moving to local machine to generate log table \\ \hline
\end{longtable}


%%manually pasted generated table for viewing on sharelatex, remove later
\begin{longtable} {|l|l|p{13cm}|}
    \hline
    \textbf{Author} & \textbf{Date} & \textbf{Message} \\ \hline
nickywongdong & 2018-04-08 & first commit \\ \hline
nickywongdong & 2018-04-10 & copying and unpacking yocto 3.19.2 image \\ \hline
nickywongdong & 2018-04-10 & copying .ext4 driver file using tar, and adding gitignore for the file \\ \hline
nickywongdong & 2018-04-10 & adding a gitignore in the hw directory as well \\ \hline
nickywongdong & 2018-04-12 & unpacking initial kernel from repo \\ \hline
nickywongdong & 2018-04-12 & adding custom version tag to Makefile under extra version \\ \hline
nickywongdong & 2018-04-12 & copying config from /scratch/files and renaming to .config, then running make menuconfig (don't know if we need this step or not) \\ \hline
nickywongdong & 2018-04-12 & moving into one git repo after build of final kernel. Adding initial state of writeup for latex \\ \hline
nickywongdong & 2018-04-12 & attempt to merge intial git with new repo \\ \hline
nickywongdong & 2018-04-12 & moving kernel outside of hw directories \\ \hline
nickywongdong & 2018-04-12 & finalizing moving kernel root tree outside of hw directory \\ \hline
nickywongdong & 2018-04-12 & adding git log latex table generator \\ \hline
nickywongdong & 2018-04-12 & changing git log table generator, previous required certain installation and rights \\ \hline
nickywongdong & 2018-04-12 & updating gitignore, removing previous latex-git-log \\ \hline
nickywongdong & 2018-04-12 & moving to local machine to generate log table \\ \hline
\end{longtable}


\section{Work Log}
\begin{tabular}{|l|l|}
\hline
     April 9 & Initial build of kernel using 3.19\\\hline
     April 10 & Running qemu with provided images for testing\\\hline
     April 10 & Rebuild with 3.19.2 to include logs, and to use config file\\\hline
     April 10 & Initial run of qemu with built kernel\\\hline
     April 12 & Removal of build, building with team\\\hline
     April 12 & Attempt to add permissions for directory\\\hline
     April 12 & Testing kernel build for another user in group directory (sekmo)\\\hline
     April 12 & Kernel build and run unsucessful, attempting to rebuild under initial user (wongnich)\\\hline
     April 12 & Modifying config, and Makefile to include custom system version tag\\\hline
     April 12 & Moving kernel build of second user to -OLD, permissions not set correctly for removal\\\hline
     April 12 & Rebuild of kernel to include custom tags, and git logs within linux root tree\\\hline
     April 12 & Attempt to run permissions script provided for teammates\\\hline
\end{tabular}
\end{document}

\documentclass[onecolumn, draftclsnofoot,10pt, compsoc]{IEEEtran}
\usepackage{graphicx}
\usepackage{url}
\usepackage{setspace}
\usepackage{tipa}
\usepackage{array, longtable}
\usepackage{listings}%
\usepackage{color}
\usepackage[usenames,dvipsnames,svgnames,table]{xcolor}
\makeatletter
\def\xfoo#1#2{\@tempcnta=0%
  \@tfor\xx:=#2\do{\advance\@tempcnta 1%
    \xx\ifnum\the\@tempcnta=#1\newline\@tempcnta=0\fi%
  }%
}

\lstset{%
  basicstyle=\ttfamily\footnotesize,
  backgroundcolor=\color{gray!10},%
  tabsize=3,
  language=bash,
  columns=fullflexible,
  %frame=shadowbox,
  frame=leftline,
  rulesepcolor=\color{gray},
  xleftmargin=20pt,
  framexleftmargin=15pt,
  keywordstyle=\color{black}\bf,
  commentstyle=\color{OliveGreen},
  %stringstyle=\color{red},
  breaklines=true,
  %numbers=left,
  %numberstyle=\tiny,
  %numbersep=5pt,
  showstringspaces=false,
  postbreak=\mbox{{/}\space},
  emph={str},emphstyle={\color{magenta}},
  alsoletter=-,%to allow qemu-system...
  %add extra keywords here.
  emph={%  
    cd, ls, cp, qemu-system-i386, mv, mkdir, make, git, gdb, uname, tar, modprobe, echo, cat, sync, dd, dmesg, diff, insmod, mkfs, ext2, mount, chmod%
    },emphstyle={\color{black}\bf},
  %remove keywords below
  deletekeywords={if, enable, continue, test, set, for, printf}
}%


\usepackage{geometry}
\geometry{textheight=9.5in, textwidth=7in}

\usepackage{graphicx}
\graphicspath{{images/}}

% 1. Fill in these details
\def \TeamName{   Group 20}
\def \GroupMemberOne{     Max Moulds}
\def \GroupMemberTwo{     Monica Sek}
\def \GroupMemberThree{     Nicholas Wong}
\def \ProfessorPerson{    Dr. Kevin McGrath}


% 2. Uncomment the appropriate line below so that the document type works
\def \DocType{    HW4 Writeup
        }

\newcommand{\NameSigPair}[1]{\par
\makebox[2.75in][r]{#1} \hfil   \makebox[3.25in]{\makebox[2.25in]{\hrulefill} \hfill    \makebox[.75in]{\hrulefill}}
\par\vspace{-12pt} \textit{\tiny\noindent
\makebox[2.75in]{} \hfil    \makebox[3.25in]{\makebox[2.25in][r]{Signature} \hfill  \makebox[.75in][r]{Date}}}}
% 3. If the document is not to be signed, uncomment the RENEWcommand below
%\renewcommand{\NameSigPair}[1]{#1}

%%%%%%%%%%%%%%%%%%%%%%%%%%%%%%%%%%%%%%%
\begin{document}
\begin{titlepage}
    \pagenumbering{gobble}
    \begin{singlespace}
        \hfill
        \par\vspace{.2in}
        \centering
        \scshape{
            \huge Operating Systems II \DocType \par
            {\large\today}\par
            \vspace{.5in}
            \vfill
            \vspace{5pt}
            %{\Large\NameSigPair{\ProfessorPerson}\par}
            {\large Prepared for }\par
            \ProfessorPerson\par
            {\large Prepared by }\par
            \GroupMemberOne\par
            \GroupMemberTwo\par
            \GroupMemberThree\par
            \vspace{5pt}
            \vspace{20pt}
        }
        \begin{abstract}
        %6. Fill in your abstract
          This document is the writeup for homework 4 of Operating Systems II Spring term 2018, written by Group 20.
          \end{abstract}
    \end{singlespace}
\end{titlepage}
\newpage
\pagenumbering{arabic}
\tableofcontents
% 7. uncomment this (if applicable). Consider adding a page break.
%\listoffigures
%\listoftables
\clearpage

\section{Design Plan}
The current algorithm for the slob.c is a first-fit algorithm, which simply allocates the first spot that is open. To implement a best-fit algorithm, we will simply need to add a checking loop before the memory is allocated, to test if that allocation is truly the best fit or not. In order to accomplish this, we can create a temporary variable which will hold our best allocation size. We can then iterate through all available spots, while testing if that spot has the smallest difference or not. Once we find it, we will continue with the first-fit algorithm and allow it to allocate that memory.

%\newpage

\section{Version Control Log}

%% uncomment this when you have the file on machine
%%\begin{longtable} {|l|l|l|}
    \hline
    \textbf{Author} & \textbf{Date} & \textbf{Message} \ \hline
nickywongdong & 2018-04-08 & first commit \\ \hline
nickywongdong & 2018-04-10 & copying and unpacking yocto 3.19.2 image \\ \hline
nickywongdong & 2018-04-10 & copying .ext4 driver file using tar, and adding gitignore for the file \\ \hline
nickywongdong & 2018-04-10 & adding a gitignore in the hw directory as well \\ \hline
nickywongdong & 2018-04-12 & unpacking initial kernel from repo \\ \hline
nickywongdong & 2018-04-12 & adding custom version tag to Makefile under extra version \\ \hline
nickywongdong & 2018-04-12 & copying config from /scratch/files and renaming to .config, then running make menuconfig (don't know if we need this step or not) \\ \hline
nickywongdong & 2018-04-12 & moving into one git repo after build of final kernel. Adding initial state of writeup for latex \\ \hline
nickywongdong & 2018-04-12 & attempt to merge intial git with new repo \\ \hline
nickywongdong & 2018-04-12 & moving kernel outside of hw directories \\ \hline
nickywongdong & 2018-04-12 & finalizing moving kernel root tree outside of hw directory \\ \hline
nickywongdong & 2018-04-12 & adding git log latex table generator \\ \hline
nickywongdong & 2018-04-12 & changing git log table generator, previous required certain installation and rights \\ \hline
nickywongdong & 2018-04-12 & updating gitignore, removing previous latex-git-log \\ \hline
nickywongdong & 2018-04-12 & moving to local machine to generate log table \\ \hline
\end{longtable}


%%manually pasted generated table for viewing on sharelatex, remove later
\begin{longtable} {|l|l|p{13cm}|}
    \hline
    \textbf{Author} & \textbf{Date} & \textbf{Message} \\ \hline
nickywongdong & 2018-06-07 & adding the initial design plan, and additional files/directories \\ \hline
nickywongdong & 2018-06-07 & initial attempt at best fit slob \\ \hline
nickywongdong & 2018-06-07 & adding program to test for output \\ \hline
nickywongdong & 2018-06-08 & forgot to select Slob.c in config. Trying to run tests on original slob.c for Fragmentation results using syscalls \\ \hline
nickywongdong & 2018-06-08 & adding necessary syscalls and backing up related files \\ \hline
nickywongdong & 2018-06-08 & adding linkage for syscalls in slob.c \\ \hline
nickywongdong & 2018-06-08 & attempting to loop through to find best fit using a minimum value and continually updating it when we find a better delta \\ \hline
nickywongdong & 2018-06-08 & need to make best a pointer \\ \hline
nickywongdong & 2018-06-08 & maybe need to define an entrypoint for for loop \\ \hline
nickywongdong & 2018-06-08 & stuck on boot, attempting to run original slob.c with test.c to see output \\ \hline
nickywongdong & 2018-06-08 & more attempts at implementing best fit algorithm \\ \hline
nickywongdong & 2018-06-08 & finished running best-fit algorithm test.c \\ \hline
nickywongdong & 2018-06-08 & updating with best-fit slob.c \\ \hline
nickywongdong & 2018-06-08 & add patch file \\ \hline

\end{longtable}


\section{Work Log}
\begin{tabular}{|l|l|}
\hline
     June 7 & Began assignment, researching from the description, and pulling examples from online sources\\\hline
     June 7 & Began design, logged with design.md\\\hline
     June 8 & Began implementation of best fit algorithm, and testing programs for Fragmentation\\\hline
     June 8 & Write up and review \\ \hline

\end{tabular}

\newpage

\section{Additional Questions}
\begin{enumerate}
    \item What do you think the main point of the assignment is?
    
     \begin{singlespace}
  
  This assignment further familiarize oneself with kernel development specifically memory management. It also introduces memory management algorithms and explores the strengths and weaknesses of each in terms of Fragmentation. 
  
    \end{singlespace}
    
    \item How did you personally approach the problem? Design the decisions, algorithm, etc.
    \begin{singlespace} 
   
   First we began a design plan for how to implement the best-fit algorithm within slob.c. After looking at examples of the differences between the original first-fit algorithm, and the best-fit algorithm, we began to implement the design plan onto slob.c. Diving right into the slob.c file was the easiest way to start. Then adding some syscalls to test was the next step. After that hurdle the project seemed to get easier. 
    
    \end{singlespace}
    
    \item How did you ensure your solution was correct? Testing details, for instance.
    \begin{singlespace} 
 
 We used a calculation for fragmentation by creating syscalls which would allow us to determine the amount of memory claimed, and freed. We ran this testing program with the kernel while it was using the first-fit slob.c, and while it was running the best-fit slob.c. We noticed a significant difference in Fragmentation.
 
 The following test program used to determine Fragmentation:
     \end{singlespace}
 \begin{lstlisting}
  for (i = 0; i < 5; i++) 
    Fragmentation = (float)freeMemory / (float)claimedMemory;
    printf("Claimed Memory: %lu\n", claimedMemory);
    printf("Free Memory: %lu\n", freeMemory);
    printf("Fragmentation: %f\n", Fragmentation);
    printf("-----\n");
    sleep(1);
  \end{lstlisting}

Testing output (first fit - original slob.c):
    \begin{lstlisting}
    Running 5 tests:
    Claimed Memory: 5543936
    Free Memory: 395614
    Fragmentation: 0.071406
    -----
    Claimed Memory: 5543936
    Free Memory: 396168
    Fragmentation: 0.071460
    -----
    Claimed Memory: 5543936
    Free Memory: 396168
    Fragmentation: 0.071460
    -----
    Claimed Memory: 5543936
    Free Memory: 396500
    Fragmentation: 0.071520
    -----
    Claimed Memory: 5543936
    Free Memory: 396500
    Fragmentation: 0.071520
    -----
  \end{lstlisting}
  
\newpage
Testing output (best fit - updated slob.c)  
  
  \begin{lstlisting}
    Running 5 tests:
    Claimed Memory: 5396480
    Free Memory: 241057
    Fragmentation: 0.044717
    -----
    Claimed Memory: 5396480
    Free Memory: 241443
    Fragmentation: 0.044741
    -----
    Claimed Memory: 5396480
    Free Memory: 241443
    Fragmentation: 0.044741
    -----
    Claimed Memory: 5396480
    Free Memory: 241443
    Fragmentation: 0.044741
    -----
    Claimed Memory: 5396480
    Free Memory: 241539
    Fragmentation: 0.044759
    -----
  \end{lstlisting}

    \item What did you learn?
    \begin{singlespace} 

 In addition to learning more about kernel development and memory management tools that helped produce usable information between git commits helped identify issues with syscalls and struct name changes over time. It also helped create a more understandable representation of the actual changes needed to implement a best fit algorithm over the existing slob code. 
 
\end{singlespace}
    
    \item How should the TA evaluate your work? Provide detailed steps to prove correctness
    
    \begin{singlespace}
  First, gather the required file(s) and a vanilla kernel of version 3.19.2. Apply the patch file which is located at:
    \end{singlespace}
    
    \begin{lstlisting}
    https://raw.githubusercontent.com/nickywongdong/CS444_GROUP20/master/hw4/hw4.patch
    \end{lstlisting}
    
    \begin{singlespace}
    Once the patch is applied and the configuration files have been loaded. (Enable slob by going into General setup and then Configure standard kernel features and finally enable slob.)
    \quad Compile the kernel. Run the vm. Then compile the testing program named test.c. Running the compiled executable should give the test results for the update slob.c file containing the best fit algorithm. To test the standard algorithm repeat all processes excluding applying the patch.
    \end{singlespace}

    And that concludes our writeup for HW4. Thank you. 
\end{enumerate}

%\section{References}
%\bibliographystyle{IEEEtran}  
%\bibliography{CS444_project4_20}

\end{document}
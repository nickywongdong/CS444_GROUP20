\documentclass[onecolumn, draftclsnofoot,10pt, compsoc]{IEEEtran}
\usepackage{graphicx}
\usepackage{url}
\usepackage{setspace}
\usepackage{tipa}
\usepackage{array, longtable}
\usepackage{listings}%
\usepackage{color}
\usepackage[usenames,dvipsnames,svgnames,table]{xcolor}
\makeatletter
\def\xfoo#1#2{\@tempcnta=0%
  \@tfor\xx:=#2\do{\advance\@tempcnta 1%
    \xx\ifnum\the\@tempcnta=#1\newline\@tempcnta=0\fi%
  }%
}

\lstset{%
  basicstyle=\ttfamily\footnotesize,
  backgroundcolor=\color{gray!10},%
  tabsize=3,
  language=bash,
  columns=fullflexible,
  %frame=shadowbox,
  frame=leftline,
  rulesepcolor=\color{gray},
  xleftmargin=20pt,
  framexleftmargin=15pt,
  keywordstyle=\color{black}\bf,
  commentstyle=\color{OliveGreen},
  %stringstyle=\color{red},
  breaklines=true,
  %numbers=left,
  %numberstyle=\tiny,
  %numbersep=5pt,
  showstringspaces=false,
  postbreak=\mbox{{/}\space},
  emph={str},emphstyle={\color{magenta}},
  alsoletter=-,%to allow qemu-system...
  %add extra keywords here.
  emph={%  
    cd, ls, cp, qemu-system-i386, mv, mkdir, make, for, git, gdb, uname, tar, modprobe, echo, cat, sync, dd, dmesg, diff, insmod, mkfs, ext2, mount, chmod%
    },emphstyle={\color{black}\bf},
  %remove keywords below
  deletekeywords={if, enable, continue, test, set}
}%


\usepackage{geometry}
\geometry{textheight=9.5in, textwidth=7in}

\usepackage{graphicx}
\graphicspath{{images/}}

% 1. Fill in these details
\def \TeamName{   Group 20}
\def \GroupMemberOne{     Max Moulds}
\def \GroupMemberTwo{     Monica Sek}
\def \GroupMemberThree{     Nicholas Wong}
\def \ProfessorPerson{    Dr. Kevin McGrath}


% 2. Uncomment the appropriate line below so that the document type works
\def \DocType{    HW3 Writeup
        }

\newcommand{\NameSigPair}[1]{\par
\makebox[2.75in][r]{#1} \hfil   \makebox[3.25in]{\makebox[2.25in]{\hrulefill} \hfill    \makebox[.75in]{\hrulefill}}
\par\vspace{-12pt} \textit{\tiny\noindent
\makebox[2.75in]{} \hfil    \makebox[3.25in]{\makebox[2.25in][r]{Signature} \hfill  \makebox[.75in][r]{Date}}}}
% 3. If the document is not to be signed, uncomment the RENEWcommand below
%\renewcommand{\NameSigPair}[1]{#1}

%%%%%%%%%%%%%%%%%%%%%%%%%%%%%%%%%%%%%%%
\begin{document}
\begin{titlepage}
    \pagenumbering{gobble}
    \begin{singlespace}
        \hfill
        \par\vspace{.2in}
        \centering
        \scshape{
            \huge Operating Systems II \DocType \par
            {\large\today}\par
            \vspace{.5in}
            \vfill
            \vspace{5pt}
            %{\Large\NameSigPair{\ProfessorPerson}\par}
            {\large Prepared for }\par
            \ProfessorPerson\par
            {\large Prepared by }\par
            \GroupMemberOne\par
            \GroupMemberTwo\par
            \GroupMemberThree\par
            \vspace{5pt}
            \vspace{20pt}
        }
        \begin{abstract}
        %6. Fill in your abstract
          This document is the writeup for homework 3 of Operating Systems II Spring term 2018, written by Group 20.
          \end{abstract}
    \end{singlespace}
\end{titlepage}
\newpage
\pagenumbering{arabic}
\tableofcontents
% 7. uncomment this (if applicable). Consider adding a page break.
%\listoffigures
%\listoftables
\clearpage

\section{Design Plan}
For this assignment we were tasked with writing a RAM Disk device driver for the Linux Kernel which should allocate a chunk of memory and present it as a block device. This requires the use of the Linux Kernel's Crypto API to add encryption to the created block device such that the device driver encrypts and decrypts data when it is written and read.

Having created a filesystem in memory before we began from that shared knowledge by researching the difference between what was done before and what is being asked in this assignment \cite{jamescoyle}. Using the logical progression from creating an unecrypted block device residing entirely in memory first and then testing to then adding a level of encryption to that device. The last element, encryption, is where the majority of the work and research was needed. 

We used the example block device given in the assignment description and extended encryption to operations that use that device. We included the crypto header, set a 16 digit base secret key and compiled our block device a module in a similar manner to assignment 2. 

Next we had to implement a cipher at a low level. In the simple block device code when a write or read was required we changed the unencrypted function logic to support our encryption scheme using the crypto api. This created a transformation of the data being written or read. We used a call to the crypto library function crypto\textunderscore cipher\textunderscore encrypt\textunderscore one for writes crypto\textunderscore cipher\textunderscore decrypt\textunderscore one for reads. 

This was built out and tested using printk statements to check the changes in situ. Further explanation of the testing and verification processes used is described in the following sections. 


%\newpage

\section{Version Control Log}

%% uncomment this when you have the file on machine
%%\begin{longtable} {|l|l|l|}
    \hline
    \textbf{Author} & \textbf{Date} & \textbf{Message} \ \hline
nickywongdong & 2018-04-08 & first commit \\ \hline
nickywongdong & 2018-04-10 & copying and unpacking yocto 3.19.2 image \\ \hline
nickywongdong & 2018-04-10 & copying .ext4 driver file using tar, and adding gitignore for the file \\ \hline
nickywongdong & 2018-04-10 & adding a gitignore in the hw directory as well \\ \hline
nickywongdong & 2018-04-12 & unpacking initial kernel from repo \\ \hline
nickywongdong & 2018-04-12 & adding custom version tag to Makefile under extra version \\ \hline
nickywongdong & 2018-04-12 & copying config from /scratch/files and renaming to .config, then running make menuconfig (don't know if we need this step or not) \\ \hline
nickywongdong & 2018-04-12 & moving into one git repo after build of final kernel. Adding initial state of writeup for latex \\ \hline
nickywongdong & 2018-04-12 & attempt to merge intial git with new repo \\ \hline
nickywongdong & 2018-04-12 & moving kernel outside of hw directories \\ \hline
nickywongdong & 2018-04-12 & finalizing moving kernel root tree outside of hw directory \\ \hline
nickywongdong & 2018-04-12 & adding git log latex table generator \\ \hline
nickywongdong & 2018-04-12 & changing git log table generator, previous required certain installation and rights \\ \hline
nickywongdong & 2018-04-12 & updating gitignore, removing previous latex-git-log \\ \hline
nickywongdong & 2018-04-12 & moving to local machine to generate log table \\ \hline
\end{longtable}


%%manually pasted generated table for viewing on sharelatex, remove later
\begin{longtable} {|l|l|p{13cm}|}
    \hline
    \textbf{Author} & \textbf{Date} & \textbf{Message} \\ \hline
nickywongdong & 2018-05-23 & uploading assignment description \\ \hline
nickywongdong & 2018-05-23 & downloading clean linux image \\ \hline
nickywongdong & 2018-05-23 & uploading initial related files \\ \hline
nickywongdong & 2018-05-23 & updating readme unordered list \\ \hline
nickywongdong & 2018-05-23 & updating ordered sublist items \\ \hline
nickywongdong & 2018-05-23 & updating readme again \\ \hline
nickywongdong & 2018-05-23 & once again updating readme \\ \hline
nickywongdong & 2018-05-23 & adding initial simple block device from example online source \\ \hline
nickywongdong & 2018-05-23 & testing initial simple block device without adding crypto \\ \hline
nickywongdong & 2018-05-23 & adding crypto parts to encrypt/decrypt data \\ \hline
nickywongdong & 2018-05-23 & testing some more crypto gegeneration \\ \hline
nickywongdong & 2018-05-24 & adding specific qemu command used for easy access \\ \hline
nickywongdong & 2018-05-24 & updating sbd to one hopefully compatible with kernel \\ \hline
nickywongdong & 2018-05-24 & testing with same crypto from last version of sbd.c \\ \hline
nickywongdong & 2018-05-24 & adding req->buffer to use bio\_data() found in swim.c -- Must be update for kernel version \\ \hline
nickywongdong & 2018-05-24 & successful compilation, plan to run with qemu to test later \\ \hline
nickywongdong & 2018-05-24 & updating printk statements for correctness \\ \hline
nickywongdong & 2018-05-24 & changing module author, and running ac1\_open script for linux kernel, and hw3 directory for teammates \\ \hline
nickywongdong & 2018-05-24 & adding command used for scp, and organized previous commands \\ \hline
nickywongdong & 2018-05-24 & adding commands used within qemu vitual machine to test kernel module \\ \hline
nickywongdong & 2018-05-25 & adding patch file \\ \hline
nickywongdong & 2018-05-25 & updating patch \\ \hline
nickywongdong & 2018-05-25 & adding block device in Kconfig \\ \hline
nickywongdong & 2018-05-25 & updating patch \\ \hline
nickywongdong & 2018-05-25 & updating patch \\ \hline
nickywongdong & 2018-05-25 & finalizing \\ \hline

\end{longtable}


\section{Work Log}
\begin{tabular}{|l|l|}
\hline
     May 23 & Began assignment by researching from the assignment description, and pulling examples from online sources\\\hline
     May 24 & Began development of crypto portion of the assigment\\\hline
     May 25 & Finalizing assignment with patch file\\\hline
     May 27 & Write up and review \\ \hline

\end{tabular}

\newpage

\section{Additional Questions}
\begin{enumerate}
    \item What do you think the main point of the assignment is?
    
     \begin{singlespace}
     In the context of operating systems and the components that provide upper level functionality this assignment requires familiarization with Linux block devices and crypto apis within a kernel. Within the context of development of a low level system component this assignment teaches one to deal with external documentation and using other external libraries to create a set of required functionalities. In the context of software development this is useful because it teaches one to develop in a way that others can use it. While this last point may be a stretch it is undoubtedly important if one plans to develop software used by others. 
 
    \end{singlespace}
    
    \item How did you personally approach the problem? Design the decisions, algorithm, etc.
    \begin{singlespace} 
    As mentioned previously the assignment was approached by building off previous knowledge gained from both previous assignments and experience outside of the class. We also relied on advice from the professor and from discussion with others familiar with the assignment goals. 

    
    \end{singlespace}
    
    \item How did you ensure your solution was correct? Testing details, for instance.
    \begin{singlespace} 
    To ensure correctness of our solution, we first made sure that the module was compiled and the .ko file existed after the make command.
    This file should be located in driver/block/ directory within the Yocto root directory.
    We then started up the virtual machine with the command similar to the one in the previous assignment, but this time removing the flag -net:none.
    This will allow us to scp the .ko file from our engr group directory into localhost.
    We then loaded the mod with insmod. The block device then showed up under the /dev directory and it was capable of being mounted to the system with mkfs.ext2.
    This caused the initialization to run within the module and demonstrate writing/reading with an additional crypto encryption/encryption process.

 
    \end{singlespace}
    
    \item What did you learn?
    \begin{singlespace} 
    While continuing to cement our understanding and comfort with developing Yocto kernel components and procedures we extended our general understanding of low level system components and their interaction with other kernel subsystems. This is most evident in the interaction between the kernel module we implemented to provide the block device and the corresponding functions used within that code that in turn used the already existing API to provide our needed cryptographic functionality.
    
   \quad Implementation alone continued lessons begun in past assignments, this assignment brought about another element that forced us to learn. Testing and verifying our solution was not as straightforward as using the implemented changes and inspecting the output and led to greater understanding of development. Sometimes verification of a solution is tougher processes than the finding the initial solution itself. Additionally, cryptography has a higher degree of required verification since ramifications from a failure in a cryptographic implementation carry different ramifications from those in the past two assignments which mainly impact system performance relating to physical device such as a hard drive and the physical or operational damages from improper implementation. The impacts from improperly verified cryptographic functionality is complete loss of the functionality and intended service. That is a crytpographic integrity is either complete or non-existant whereas a poorly implemented seek algorithm for a hard drive device can still provide some level of service. 


    \end{singlespace}
    \item How should the TA evaluate your work? Provide detailed steps to prove correctness
    \begin{singlespace}
    Start with the 3.19.2 Yocto kernel source and our project submission and enter into that submission directory. Use the patch file given with this write up or located at 
    \end{singlespace}
    \begin{lstlisting}
    https://github.com/nickywongdong/CS444_GROUP20/blob/master/hw3/patch/hw3.patch
    \end{lstlisting}
    \subsection*{Applying the patch and compiling the kernel image}
    \begin{singlespace}
    First, please apply the patch supplied to the v3.19.2 branch of the Yocto Kernel, and while in that directory check that patch applied by looking for the simple device block driver. Use the processes described in past assignments.

    Once compilation has completed, there should be a kernel image file in the x86 directory and subdirectory boot. Please start the VM using the following command changed to reflect booting from that image:
    \end{singlespace}
    \begin{lstlisting}
    > qemu-system-i386 -gdb tcp::5520 -S -nographic -kernel linux-yocto-3.19.2/arch/x86/boot/bzImage -drive file=core-image-lsb-sdk-qemux86.ext4,if=virtio -enable-kvm -usb -localtime --no-reboot --append "root=/dev/vda rw console=ttyS0 debug"
    \end{lstlisting}
    \begin{singlespace}
    Now that the virtual machine is running transfer the compiled kernel object file  compiled and use insmod instead of modprobe to load the device driver. Once loaded enable it with
    \end{singlespace}
    \begin{lstlisting}
    > insmod sdb.ko
    > mkfs.ext2 /dev/sbd0
    > mount /dev/sbd0 /test
    \end{lstlisting}
    This should create a filesystem on that block device and that filesystem should be encrypted. Generate file I/O and view dmesg or one may need to cat the contents of /var/log/dmesg or /var/log/kernel for confirmation from the encryption operation described earlier in the document. Below is an example output from those functions:
    
    Output after mounting ext2 filesytem on block device:
    \begin{lstlisting}
    [ simple_blk_dev.c: sbd_transfer() ] - READ Transferring Data
    [ simple_blk_dev.c: sbd_transfer() ] - Crypto key is set and encrypted
    [ simple_blk_dev.c: sbd_transfer() ] - Read 4096 bytes to device data
    [ simple_blk_dev.c: sbd_transfer() ] - UNENCRYPTED DATA VIEW:
    11218216925422555629918715122919119396021116716313918237209177622711212724310125
    03119916716313918237209177622711212724310125031199209291361421081062391354813514
    81854735130771139544212208433901297415916522357451541139544212208433901297415916
    52235745154113954421
    [ simple_blk_dev.c: sbd_transfer() ] - ENCRYPTED DATA VIEW:
    25525525570000000000000000000000000000000000000000000000000000000000012825525525
    52552552552552552552552552552552552552552552552552552552552552552552552552552552
    55255255255255255255
    [ simple_blk_dev.c: sbd_transfer() ] - Transfer and Encryption Completed
    [ simple_blk_dev.c: sbd_transfer() ] - READ Transferring Data
    [ simple_blk_dev.c: sbd_transfer() ] - Crypto key is set and encrypted
    [ simple_blk_dev.c: sbd_transfer() ] - Read 4096 bytes to device data
    [ simple_blk_dev.c: sbd_transfer() ] - UNENCRYPTED DATA VIEW:
    16716313918237209177622711212724310125031199167163139182372091776227112127243101
    25031199167163139182372091776227112127243101250311991671631391823720917762271121
    27243101250311991671631391823720917762271121272430125031199167163139182372091776
    2271121272431012503119916716313918
    [ simple_blk_dev.c: sbd_transfer() ] - ENCRYPTED DATA VIEW:
    00000000000000000000000000000000000000000000000000000000000000000000000000000000
    0000000000000000000
    [ simple_blk_dev.c: sbd_transfer() ] - Transfer and Encryption Completed
    [ simple_blk_dev.c: sbd_transfer() ] - READ Transferring Data
    [ simple_blk_dev.c: sbd_transfer() ] - Crypto key is set and encrypted
    [ simple_blk_dev.c: sbd_transfer() ] - Read 4096 bytes to device data
    [ simple_blk_dev.c: sbd_transfer() ] - UNENCRYPTED DATA VIEW:
    16716313918237209177622711212724310125031199167163139182372091776227112127243101
    25031199167163139182372091776227112127243101250311991671631391823720917762271121
    27243101250311991671631391823720917762271121272431012503119916716313918237209177
    62271121272431012503119916716313918
    [ simple_blk_dev.c: sbd_transfer() ] - ENCRYPTED DATA VIEW:
    00000000000000000000000000000000000000000000000000000000000000000000000000000000
    00000000000000000000
    [ simple_blk_dev.c: sbd_transfer() ] - Transfer and Encryption Completed
    [ simple_blk_dev.c: sbd_transfer() ] - READ Transferring Data
    [ simple_blk_dev.c: sbd_transfer() ] - Crypto key is set and encrypted
    [ simple_blk_dev.c: sbd_transfer() ] - Read 4096 bytes to device data
    [ simple_blk_dev.c: sbd_transfer() ] - UNENCRYPTED DATA VIEW:
    18369001210019669001210020969001210022269001210023569001210024869001210057000121
    001870001210031700012100447000121007170001210083700012100957000
    [ simple_blk_dev.c: sbd_transfer() ] - ENCRYPTED DATA VIEW:
    12111142159911111618819197202461642108212226107810131150639184622152814101176194
    63139154567910917486228227114163222366915917464301911792341622120511897671912431
    98223227151912022102344023115171638985161231158471221671679881241606572412422216
    214720717190128192
    [ simple_blk_dev.c: sbd_transfer() ] - Transfer and Encryption Completed
    [ simple_blk_dev.c: sbd_transfer() ] - READ Transferring Data
    [ simple_blk_dev.c: sbd_transfer() ] - Crypto key is set and encrypted
    [ simple_blk_dev.c: sbd_transfer() ] - Read 4096 bytes to device data
    [ simple_blk_dev.c: sbd_transfer() ] - UNENCRYPTED DATA VIEW:
    45971181203245103324579503245115116100611031101175748324511232451021101114511511
    61141059911645971081059711510511010332451021101114599111109109111110324510211410
    11034511511611411799116451141011161171141103245102110111451121059932451021021141
    0110111511697110100105
    [ simple_blk_dev.c: sbd_transfer() ] - ENCRYPTED DATA VIEW:
    23614217105342195129172219164163652912981125246362222111037218777281809194222082
    12137524121819716015761013613823924151906118694633401682306351158132239502095119
    38125142216371481333210912493142190185622142104888195221220210223192214253105229
    581411961282057159
    [ simple_blk_dev.c: sbd_transfer() ] - Transfer and Encryption Completed
    [ simple_blk_dev.c: sbd_transfer() ] - READ Transferring Data
    [ simple_blk_dev.c: sbd_transfer() ] - Crypto key is set and encrypted
    [ simple_blk_dev.c: sbd_transfer() ] - Read 1024 bytes to device data
    [ simple_blk_dev.c: sbd_transfer() ] - UNENCRYPTED DATA VIEW:
    25019733153303123670967131243156197170737410723810622311522419221711079158017775
    82419171952021416215813675333202214784549375021117110223693214158228174157100641
    25181578161141522454209793211211176121187127165273199239160169109177107841521901
    91127160183252190
    [ simple_blk_dev.c: sbd_transfer() ] - ENCRYPTED DATA VIEW:
    64000020025000228100530001000000000000320003200640000000130691191002552558323910
    1000130691191000000001000000011000128000560002000
    [ simple_blk_dev.c: sbd_transfer() ] - Transfer and Encryption Completed
    [ simple_blk_dev.c: sbd_transfer() ] - READ Transferring Data
    [ simple_blk_dev.c: sbd_transfer() ] - Crypto key is set and encrypted
    [ simple_blk_dev.c: sbd_transfer() ] - Read 1024 bytes to device data
    [ simple_blk_dev.c: sbd_transfer() ] - UNENCRYPTED DATA VIEW:
    18222622855456237721547142115291862171032118253972023524935174217106231551961951
    30167163139182372091776227112127243101250311991671631391823720917762271121272431
    01250311991671631391823720917762271121272431012503119916716313918237209177622711
    21272431012503119916716313918
    [ simple_blk_dev.c: sbd_transfer() ] - ENCRYPTED DATA VIEW:
    40005000600022815302000000000000000000000000000000000000000000000000000000000000
    00000000000000000000000
    [ simple_blk_dev.c: sbd_transfer() ] - Transfer and Encryption Completed
    [ simple_blk_dev.c: sbd_transfer() ] - READ Transferring Data
    [ simple_blk_dev.c: sbd_transfer() ] - Crypto key is set and encrypted
    [ simple_blk_dev.c: sbd_transfer() ] - Read 1024 bytes to device data
    [ simple_blk_dev.c: sbd_transfer() ] - UNENCRYPTED DATA VIEW:
    21416420515823784337702321092337120123911017923923225311110981281235411121113249
    36191167163139182372091776227112127243101250311991671631391823720917762271121272
    43101250311991671631391823720917762271121272431012503119916716313918237209177622
    71121272431012503119916716313918
    [ simple_blk_dev.c: sbd_transfer() ] - ENCRYPTED DATA VIEW:
    00000000130691191130691191130691191000000000000000000000000000000000000000000000
    00000000000000000000000000000000000
    [ simple_blk_dev.c: sbd_transfer() ] - Transfer and Encryption Completed
    [ simple_blk_dev.c: sbd_transfer() ] - WRITE Transferring Data
    [ simple_blk_dev.c: sbd_transfer() ] - Crypto key is set and encrypted
    [ simple_blk_dev.c: sbd_transfer() ] - Write 1024 bytes to device data
    [ simple_blk_dev.c: sbd_transfer() ] - UNENCRYPTED DATA VIEW:
    25019733153303123670967131243156197170737410723810622311522419221711079158017775
    82419171952021416215813675333202214784520512044831804813622223411685223249220910
    18157816114152245420979321121117612118712716527319923916016910917710784152190191
    127160183252190
    [ simple_blk_dev.c: sbd_transfer() ] - ENCRYPTED DATA VIEW:
    64000020025000228100530001000000000000320003200640000000150691191102552558323900
    1000130691191000000001000000011000128000560002000
    [ simple_blk_dev.c: sbd_transfer() ] - Transfer and Encryption Completed
\end{lstlisting}

    And that concludes our writeup for HW3. Thank you. 
  
\end{enumerate}


\section{References}
\bibliographystyle{IEEEtran}  
\bibliography{CS444_project3_20}

\end{document}